\chapter{Appendix}
\label{cha:appendix}

\section{Reaction coordinates}
\label{sec:reaction coordinates}

In the present work valance angles and torsion angles were used as collective variables for Umbrella Sampling, metaD, ABF, eABF or meta-eABF simulations. Below analytic expressions for the calculation of both variables, their gradients, inverse gradients and divergence of inverse gradients are given.

\subsection{Bond Angle}
The bond angle $\theta$ between three Cartesian coordinates ($r_1, r_2, r_3$) is given by the dot product between bond distances $r_{ij}=r_j - r_i$:
\begin{equation}
  \cos \theta = \hat{r}_{21} \cdot \hat{r}_{23}
\end{equation}
The derivative of $\theta$ in Cartesian coordinates is then given by
\begin{equation}
  \frac{\partial \theta}{\partial r_1} = \frac{norm(\hat{r}_{21} \times (\hat{r}_{21} \times \hat{r}_{23}))}{|r_{12}|}
\end{equation}
\begin{equation}
  \frac{\partial \theta}{\partial r_3} = \frac{norm(\hat{r}_{32} \times (\hat{r}_{21} \times \hat{r}_{23}))}{|r_{23}|}
\end{equation}
\begin{equation}
  \frac{\partial \theta}{\partial r_2}= -\frac{\partial \theta}{\partial r_1} - \frac{\partial \theta}{\partial r_3}
\end{equation}
Choosing as inverse gradient $\textbf{v}_i = \nabla \theta/|\nabla \theta|^2$ the divergence of vector field $\textbf{v}_i$ is given by:
\begin{equation}
  \nabla \cdot \textbf{v}_i = \cot \theta
\end{equation}

\subsection{Torsion Angle}

The torsion angle $\phi$ between four Cartesian coordinates ($r_1, r_2, r_3, r_4$) is given by:
\begin{equation}
  \tan \phi = \frac{(\hat{r}_{12} \times n_1) \cdot n_2}{n_1 \cdot n_2}
\end{equation}
with
\begin{equation}
  n_1 = r_{23} - (r_{12} \cdot \hat{r}_{23})\hat{r}_{23}
\end{equation}
\begin{equation}
  n_2 = r_{34} - (r_{34} \cdot \hat{r}_{23})\hat{r}_{23}
\end{equation}
The derivative of $\phi$ in Cartesian coordinates is given by:
\begin{equation}
  \frac{\partial\phi}{\partial r_1} = - \frac{\hat{r}_{21}\times\hat{r}_{23}}{|r_{12}|\sin^2\theta_{123}}
\end{equation}
\begin{equation}
  \frac{\partial\phi}{\partial r_4} = - \frac{\hat{r}_{34}\times\hat{r}_{23}}{|r_{34}|\sin^2\theta_{234}}
\end{equation}
\begin{equation}
  \frac{\partial\phi}{\partial r_2} = c_{123}\frac{\partial\phi}{\partial r_1} - b_{432}\frac{\partial\phi}{\partial r_4}
\end{equation}
\begin{equation}
  \frac{\partial\phi}{\partial r_3} = - \frac{\partial\phi}{\partial r_1} - \frac{\partial\phi}{\partial r_4} - \frac{\partial\phi}{\partial r_2}
\end{equation}
with
\begin{equation}
  c_{123} = \frac{|r_{12}|\cos\theta_{123}}{|r_{23}|}-1
\end{equation}
\begin{equation}
  b_{432} = \frac{|r_{34}|\cos\theta_{234}}{|r_{23}|}
\end{equation}
Choosing as inverse gradient $\textbf{v}_i = \nabla \phi/|\nabla \phi|^2$ the divergence of vector field $\textbf{v}_i$ is zero:
\begin{equation}
  \nabla \cdot \textbf{v}_i = 0
\end{equation}
