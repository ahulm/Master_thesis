\documentclass[
	12pt,                      % Schriftgroesse 12pt
	a4paper,                   % Layout fuer Din A4
	english,                   % neue deutsche Rechtschreibung nach der Reform
	oneside,                   % Layout fuer einseitigen Druck
	headinclude,               % Kopfzeile wird Seiten-Layouts mit beruecksichtigt
	headsepline,               % horizontale Linie unter Kolumnentitel
	plainheadsepline,          % horizontale Linie auch beim plain-Style
	BCOR=12mm,                  % Korrektur für die Bindung
	DIV=18,                    % DIV-Wert für die Erstellung des Satzspiegels, siehe scrguide
	parskip=half,              % Absatzabstand statt Absatzeinzug
	openany,                   % Kapitel können auf geraden und ungeraden Seiten beginnen
	bibliography=totoc,        % Literaturverz. wird in und sonstige verzeichnisse mit ins Inhaltsverzeichnis
	numbers=noenddot,          % Kapitelnummern immer ohne Punkt
	captions=tableheading,     % korrekte Abstaende bei TabellenUEBERschriften3
	]{scrbook}[2001/07/30]     % scrbook-Version mind. 2.8j von 2001/07/30

%##########################################################################################################
%
% Packete laden
%
%##########################################################################################################

\usepackage[english]{babel}
\usepackage[utf8]{inputenc}                    % Input-Encoding: ansinew for Windows
\usepackage[T1]{fontenc}                       % T1-kodierte Schriften, korrekte Trennmuster für Worte mit Umlauten
\usepackage{ae}                                % Für PDF-Erstellung

\usepackage[
	format=hang,
	font={footnotesize, sf},
	labelfont={bf},
	margin=1cm,
	aboveskip=5pt,
	belowskip=5pt
	]{caption,subfig}                          % mehrzeilige Captions ausrichten; subfig: Untergrafiken
\usepackage{wrapfig}

\usepackage[centertags]{amsmath}               % AMS-Mathematik, centertags zentriert Nummer bei split
\usepackage{amssymb}                           % zusätzliche Symbole
\usepackage{trfsigns} 					       % für bestimmte Symbole
\usepackage{graphicx}                          % zum Einbinden von Grafiken
\usepackage[svgnames,table,hyperref]{xcolor}
\usepackage{float}                             % u.a. genaue Plazierung von Gleitobjekten mit H
\usepackage{epsfig}                            % eps Format für Grafiken
\usepackage[pdftex,pstarrows]{pict2e}
\usepackage{array}
\usepackage{listings}						   % Code-Einbindungsumgebung
\usepackage{courier}                           % verwende Courier statt cmtt als monospace-schrift
\usepackage{setspace}                          % Zeilenabstand einstellbar
\usepackage{rotating}
\onehalfspacing                                % eineinhalbzeilig einstellen
\usepackage{longtable}					       % Ermöglicht Tabellen die über den Seitenumbruch gehen (s. Symbolverzeichnis)
\usepackage{scrlayer-scrpage}
\usepackage{scrhack}                         % Kopf und Fusszeilen-Layout passt besser zur Dokumentklasse KOMA-Skript (scrbook) als das Pake fancyhdr, sonst ziemlich gleichwertig
\typearea[current]{current}                    % Neuberechnung des Satzspiegels mit alten Werten nach Änderung von Zeilenabstand,etc
\usepackage{xcolor,colortbl}                   % Packet um Tabellen bunt auszufüllen
\usepackage{wasysym}                           % Promillezeichen und co.
\usepackage{tabularx}
\usepackage{booktabs}
\usepackage{multirow}
\usepackage{braket}
\usepackage{algorithm}
\usepackage{algorithmic}

\usepackage[
	backend=biber,
	style=chem-angew,			% Style Angewante Chemie
]{biblatex}     		 		% Biblatex Paket für Referenzen
\addbibresource{literatur.bib}	% File were you store sources

%##########################################################################################################
%
% PDF-Erzeugung: pdflatex statt latex aufrufen!
%
%##########################################################################################################

\pdfoutput=1
\usepackage[pdftex,  % muss letztes Package sein!
	pdftitle={Perovscites for Photovoltaics},%
	pdfauthor={Andreas Hulm},%
	pdfsubject={Perovscites},%
	pdfkeywords={ ... },%
	pdfstartview={FitH},%
	pdfstartpage={5},%
	bookmarks,%
	raiselinks,%
	pageanchor,%
	hyperindex,%
	colorlinks,%
	citecolor=black!60!black,%
	linkcolor=black!70!black,%
	urlcolor=magenta!70!black,%
	filecolor=magenta!70!black,%
	menucolor=orange!70!black,%
    ]{hyperref}

%##########################################################################################################
%
% New- and Renew-Commands
%
%##########################################################################################################


\renewcommand{\headfont}{\normalfont\sffamily}             % Kolumnentitel serifenlos
\renewcommand{\pnumfont}{\normalfont\ttfamily\bfseries}    % Seitennummern typewriter und fett
\pagestyle{scrheadings}

% Einkommentieren falls beidseitige Darstellung erwünscht!!! aktuell definiert: oneside -> Layout fuer einseitigen Druck

%\ihead[]{\headmark}              % Kolumnentitel immer oben innen
%\ohead[\pagemark]{\pagemark}     % Seitennummern immer oben aussen
%\lefoot[]{}
%\rofoot[]{}                      % Seitennummern in der Fusszeile loeschen

\newcommand {\jkarray}[1]{\ensuremath{\underline{#1}}}
\newcommand {\jkmatrix}[1]{\ensuremath{\underline{\underline{#1}}}}
\newcommand {\einheit}[1]{\ensuremath{\mathrm{\left[#1\right]}}}
\newcommand {\lived}[2]{($\ast$#1, $\dagger$#2)}  %



%###########################################################################################################
%
% Parameter für die jeweiligen Packete definieren
%
%###########################################################################################################

\lstdefinestyle{cppcode}{language={[Visual]C++},%
	basicstyle=\ttfamily\footnotesize,%
	keywordstyle={\color{Navy} \bfseries},%
	identifierstyle={\color{DarkRed}},%
	commentstyle={\color{DarkOrange!50!black}\slshape},%
	stringstyle={\color{DarkGreen}},%
	showstringspaces=false,%
	backgroundcolor={\color{LightSkyBlue!40}},%
	columns=fixed,%
	keepspaces=true,%
	basewidth={0.55em},%
	frame=shadowbox,%
	rulesepcolor=\color{Gray},%
	breaklines=true,%
	numbers=left,%
	numberstyle=\tiny,%
	escapeinside={°(}{)°},%
	moredelim={[is][\bfseries]{°^}{^°}},%
	belowcaptionskip=0.5cm%
	}%

\lstdefinestyle{fort}{language={[95]Fortran},%
	basicstyle=\ttfamily\small,%
	keywordstyle={\color{Navy} \bfseries},%
	identifierstyle={\color{DarkRed}},%
	commentstyle={\color{DarkOrange!50!black}\slshape},%
	stringstyle={\color{DarkGreen}},%
	showstringspaces=false,%
	backgroundcolor={\color{LightSkyBlue!40}},%
	columns=fullflexible,%
	keepspaces=true,%
	basewidth={0.6em},%
	rulesepcolor=\color{Gray},%
	frame=shadowbox,%
	escapeinside={°(}{)°},%
	moredelim={[is][\bfseries]{°^}{^°}},%
	belowcaptionskip=0.5cm%
	}%


\lstdefinestyle{pseudocode}{basicstyle=\ttfamily\small,%
	columns=fixed,%
	keepspaces=true,%
	basewidth={0.55em},%
	frame=shadowbox,%
	backgroundcolor={\color{LightSkyBlue!40}},%
	rulesepcolor=\color{Gray},%
	escapeinside={°(}{)°},%
	moredelim={[is][\bfseries]{°^}{^°}},%
	belowcaptionskip=0.5cm%
	}%

\lstdefinestyle{maple}{%
	basicstyle=\sffamily\small\color{Red}\bfseries,%
	rulecolor=\color{Black},%
	columns=fixed,%
	keepspaces=true,%
	basewidth={0.55em},%
	frame=shadowbox,%
	numbers=left,%
	numberstyle=\tiny\color{Black},%
	numberblanklines=false,%
	rulesepcolor=\color{Gray},%
	breaklines=true,%
	breakautoindent=true,%
	backgroundcolor={\color{LightBlue!60}},%
	rulesepcolor=\color{Gray},%
	escapeinside={°(}{)°},%
	moredelim={[is][\bfseries]{°^}{^°}},%
	belowcaptionskip=0.5cm%
	}%

\lstdefinestyle{matlab}{language={Matlab},%
	basicstyle=\ttfamily\small,%
	keywordstyle={\color{Navy} \bfseries},%
	identifierstyle={\color{DarkRed}},%
	commentstyle={\color{DarkOrange!50!black}\slshape},%
	stringstyle={\color{DarkGreen}},%
	showstringspaces=false,%
	backgroundcolor={\color{LightSkyBlue!30}},%
	breaklines=true,%
	breakautoindent=true,%
	columns=fullflexible,%
	keepspaces=true,%
	basewidth={0.6em},%
	rulesepcolor=\color{Gray},%
	frame=shadowbox,%
	numbers=left,%
	numberstyle=\tiny\color{Black},%
	escapeinside={°(}{)°},%
	moredelim={[is][\bfseries]{°^}{^°}},%
	belowcaptionskip=0.5cm%
	}%

\lstdefinelanguage{Python}{
	basicstyle=\ttfamily\small,%
	keywordstyle={\color{Navy} \bfseries},%
 	keywords={typeof, null, catch, switch, in, int, str, float, self, boolean, throw, import,return, class, if ,elif, endif, while, do, else, True, False , catch, def, from, for},
 	identifierstyle=\color{black},
	comment=[l]{\#},
	commentstyle={\color{gray}\slshape},%
	stringstyle={\color{DarkGreen}},%
	backgroundcolor={\color{LightSkyBlue!30}},%
	breaklines=true,%
	breakautoindent=true,%
	columns=fullflexible,%
	keepspaces=true,%
	basewidth={0.6em},%
	rulesepcolor=\color{Gray},%
	frame=shadowbox,%
	numbers=left,%
	numberstyle=\tiny\color{Black},%
	escapeinside={°(}{)°},%
	moredelim={[is][\bfseries]{°^}{^°}},%
 	sensitive=false,
 	morecomment=[s]{/*}{*/},
	belowcaptionskip=0.5cm%
}

\graphicspath{{figs/}{bilder/}{plots/}}    % Falls texinput nicht gesetzt -> Bildverzeichnisse

% hier sind Worte zu definieren die in der Worttrennung falsch oder nicht erkannt werden!

\hyphenation{Post-pro-cess-ing--In-te-gral}


%###########################################################################################################
%
% Aufbau des Dokuments -> Einfügen der einzelnen Teile
%
%###########################################################################################################

% '''''''''''''''''''''''''''''''''''''''''''''''''''''''''''''''''''''
\newcommand{\sectionnumbering}[1]{%
	\setcounter{section}{0}%
	\renewcommand{\thesection}{\csname #1\endcsname{section}}}
% '''''''''''''''''''''''''''''''''''''''''''''''''''''''''''''''''''''

\newcounter{romanPagenumber} % neuen Seitenzähler als Variable definieren

\begin{document}
	\frontmatter

	\setheadsepline{0.0pt} 		  %Dicke der Trennlinie Kopfzeile - Text -> für Erklärung Änderungen ausschalten und erst ab Kurzzusammenfassung beginnen!

	\pagenumbering{Roman}         % romanische Nummerierung für die Deckblätter, Inhaltsverzeichnis und co.

	%
\begin{titlepage}

\begin{center}
{
%\fontsize{18}{18}\selectfont   % font Gr��e undefiniert-> es wird nur mit \text gearbeitet

\vspace*{-1.5cm}
\hfill \includegraphics*[width=3cm, keepaspectratio=true]{lmulogo.png}

\hrule                                 % horizontale Linie ein

\textsc{\LARGE Ludwig-Maximilians-University Munich}\\[2cm]

\textsc{\Large }\\[0.5cm]

\textsc{\Large Faculty of Chemistry and Pharmacy}\\[2.5cm]

\textbf{\LARGE Advanced enhanced sampling algorithms for free Energy calculation in collective variable space} \\[0.5cm]

\textsc{\Large Master Thesis in Theoretical Chemistry}\\[2.0cm]

\textsc{AK Ochsenfeld}\\[2.0cm]

\textsc{by}\\
\textsc{Andreas Valentin Hulm} \\[1,5cm]
%\textsc{geboren am 29.04.1997} \\
%\textsc{in Freising} \\ [2,5cm]

%Beginn der Bachelorarbeit: 07.05.2018 \\
%Bachelorarbeit beim Pr�fungsamt eingereicht am \today

\vspace{3.5cm}


}
\end{center}

\end{titlepage}


%\thispagestyle{empty}     % 2. Seite leer!
%\section*{}
     % Deckblatt Titel
	%\addchap{List of symbols}
\markboth{List of symbols}{List of symbols}
\label{cha:symbols}

\begin{longtable}[l]{lcp{10cm}}
	ABF		&&  Adaptive-biasing Force \\
	eABF  &&  extended Adaptive-biasing Force \\
  US    &&  Umbrella sampling \\
	WTM   &&  Well-Tempered Metadynamics \\
\end{longtable}


	\setheadsepline{0.5pt}        % Dicke der Trennlinie Kopfzeile - Text

	\tableofcontents              % Inhaltsverzeichnis

	\clearpage

	\setcounter{romanPagenumber}{\value{page}} % eigener Seitenzähler erhält aktuelle römische Seitenzahl

	\mainmatter                   % den Hauptteil beginnen
	\pagenumbering{arabic}        % ab hier wieder arabische Nummerierung

	%\chapter{Introduction}
\label{cha:introduction}
Free energy differences are the driving force of chemical processes at or near thermodynamic equilibrium and therefore the central quantity that determines the behavior of these systems.\autocite{chipot2007free}
However, the calculation of free energies still constitutes one of the major challenges of computational chemistry.\autocite{kastner2011umbrella}
This is because free energy contains entropy, which is a measure for the available space of a system.
For more than a few atoms, mapping available space requires extensive sampling, making free energy calculations computationally exceedingly demanding.\autocite{kollman1993free}
Therefore, although the statistical-mechanical foundations for the calculation of free energy curves were laid decades ago,\autocite{chipot2007free} it is only in recent years that increasing computational power combined with major advances in the efficiency of quantum-mechanical/molecular-mechanical (QM/MM) codes\autocite{ochsenfeld2007linear,shao2015advances,acun2018scalable} made their application possible.
Today free energy calculations are frequently used in several important areas ranging from biochemistry\autocite{gumbart2013standard,fu2017new,capelli2019chasing}, to pharmacology\autocite{yu2017computer,sinko2013accounting} or
nanotechnology.\autocite{fu2017lubricating,chen2019tumbling}

In practice to sample configurations of chemical systems time trajectories are calculated by means of molecular dynamics (MD) or Monte-Carlo (MC) simulations.\autocite{ponder2003force,burke2012perspective}
However, as most chemical reactions involve crossing of free energy barriers, reaction coordinates often constitute slow degrees of freedom.
This means that trajectories stay kinetically trapped in \textit{metastable} states, e.g., the educt or product state, and are unable to explore the full reaction coordinate.\autocite{chipot2007free}
%Simple trajectories are therefore poorly suited for sampling of reaction coordinates.\autocite{chipot2007free}
To address this problem the vary active research field of \textit{enhanced sampling} emerged, which has already produced numerous different approaches to speed up exploration of reaction coordinates.
\autocite{jiang2010free, sugita1999replica,den2000thermodynamic, ciccotti2005blue, barducci2008well}

One particular successful class of algorithms relies on the definition of \textit{collective variables} (CVs).
This variables need to distinguish between the educt and product states, while ideally capturing all slow degrees of freedom along the way.\autocite{fiorin2013using}
Typically maximal two dimensional variables are chosen, because of the massive growth of computational cost of sampling in higher dimensional space, also known as \textit{curse of dimensionality}.\autocite{koppen2000curse}
The potential energy or forces along the CVs are then altered in a way, that increases the time spent in important regions.
One of the oldest and most widespread approaches is \textit{Umbrella Sampling}\autocite{kastner2011umbrella}, where bias potentials along the reaction coordinate drive a system from the reactant to the product state.
The intermediate steps are covered by a series of windows, in each of which a MD simulation is performed.
From this simulations the full free energy curve can be calculated by combining all windows with the weighted histogram analysis method (WHAM).\autocite{kumar1992weighted}
This approach enables efficient sampling along the reaction coordinate due to parallelisation.
However, it also requires some knowledge of the free energy curve prior to simulation in order to adequately choose the bias potentials.
In addition, setting up and analyzing multiple MD simulations requires huge computer resources and is time consuming.

To address both shortcomings this thesis will focus on another class of enhanced sampling algorithms, termed \textit{adaptive biasing} methods.\autocite{barducci2008well,comer2015adaptive,lesage2017smoothed}
Here a time-dependent, self-learning bias potential or force is introduced, that evolves during the simulation to encourage uniform sampling along the CV.
One can think of two complementary approaches to build time-dependent biases:
The first one, e.g., \textit{metadynamics} (MtD)\autocite{barducci2011metadynamics} and its variants, encourages sampling by flooding valleys of the free energy landscape with a time-dependent potential.
Because of its simplicity and straightforward implementation it has been integrated in almost all popular MD engines and was broadly utilized for a large variety of problems.\autocite{vymetal2011gyration,tanida2020alchemical,ikeda2005hydration}
The second one, termed \textit{adaptive biasing force} (ABF)\autocite{comer2015adaptive} method, flattens the free energy landscape by application of a time-dependent bias force.
Despite its outstanding stability and beneficial formal convergence properties, the practical implementation and application of ABF has been thwarted by the analytical formulation of the biasing forces, thereby limiting the scope of the numerical scheme.\autocite{fiorin2013using}
Recently this limitations could been lifted by extended-Lagrangian based methods, where a fictitious particle is coupled to the CV.
In this framework the bias force used for enhanced sampling only acts on the fictitious particle, which renders its implementation trivial.
The resulting extended-system ABF (eABF)\autocite{lesage2017smoothed} method combines the wide applicability of MtD with the convergence properties of ABF.
Additionally the calculation of free energy curves can be separated from sampling acceleration with an asymptotically unbiased free energy gradient estimator, termed \textit{corrected z-averaged restraint} (CZAR).\autocite{lesage2017smoothed}
The resulting flexibility in the choice of bias force can be used to combine both metadynamics and ABF to well-tempered metadynamics extended-system ABF (WTM-eABF),\autocite{fu2018zooming,fu2019taming} a highly potent enhanced-sampling scheme, which stands out due to its efficiency and robustness.

In this thesis all aforementioned adaptive biasing algorithms are combined with highly efficient QM/MM calculations in the in-house FermiONs++\autocite{kussmann2013linear} program package, to enable the broad application of free energy calculations for large molecular systems at accurate level of theory.

	\chapter{Theoretical Background}
\label{cha:theory}

\section{Statistical thermodynamics and free energy differences}

The formulation of the problem of free energy estimation starts with the Hamiltonian
\begin{equation}
  \textbf{H}(\textbf{r})=\sum_{i=1}^{N}\frac{\textbf{p}_{i}^2}{2m_i} + U(\textbf{x}_1,...,\textbf{x}_N)
\end{equation}
of an arbitrary chemical system, where $\textbf{r}=(\textbf{x}_1,...,\textbf{x}_N,\textbf{p}_1,...,\textbf{p}_N)$ denotes a point in the phase space of the N-particle system of interest and $U(\textbf{x}_1,...,\textbf{x}_N)$ is the potential energy of the system given by any quantum chemical method or force field. If the canonical ensemble (an ensemble with fixed number of particles N, volume V and temperature T) of such a system is sampled, for example by means of Langevin dynamics with sufficiently soft damping and small stochastic forces, the probability distribution $\rho(\textbf{x})$ in the configuration space $\textbf{x}$ follows the Boltzmann distribution
\begin{equation}
  \rho(\textbf{x})=\frac{e^{-\beta U(\textbf{x})}}{\int e^{-\beta U(\textbf{x})} d\textbf{x}}=Q^{-1}e^{-\beta U(\textbf{x})}
\end{equation}
were $Q$ denotes the partition function and $\beta=(k_B T)^{-1}$ the inverse temperature, $k_B$ is the Boltzmann constant. The partition function is the central quantity of statistical thermodynamics, relating macroscopic thermodynamic quantities to the microscopic details of a system. The free energy difference between two states A and B is defined by the ratio of the corresponding partition functions $Q_A$ and $Q_B$.
\begin{equation}
  A = -\beta^{-1}\ln \frac{Q_A}{Q_B}
\end{equation}
This immediately shows that the challenge of calculating free energies really consists in exploring the configuration space of a system such that relevant states are adequately sampled.



\section{Adaptive Biasing Methods}
\label{sec:adaptive biasing}

Instead of dividing the reaction coordinate in several windows, with adaptive biasing methods the free energy can be estimated from one single simulation. For this purpose the systems dynamics are biased towards states corresponding to large values of the free energy along the transition coordinate via a history-dependent biasing potential. In contrast to other importance sampling strategies like umbrella sampling, this methods require no prior knowledge of the free-energy landscape at hand. Instead, the biasing potential automatically converges towards the free energy, enabling diffusive behavior along the transition coordinate.

There are multiple adaptive biasing methods available, only differing in the choice of bias. Methods based on metadynamics (metaD) disfavor already visited states by accumulating repulsive potentials along the reaction coordinate (section \ref{sec:metaD}), while adaptive biasing force (ABF) methods compensate the mean force along the reaction coordinate to obtain uniform sampling (sections \ref{sec:ABF} and \ref{sec:eABF}). Meta-eABF combines both complementary approaches to speed up the convergence of the free energy estimate (section \ref{sec:meta-eABF}).

In principle adaptive biasing methods only rely on the sampling of the canonical ensemble. One simple way to achieve this is using Langevin dynamics with sufficiently soft damping and small stochastic forces. A schematic procedure of adaptively biased Langevin dynamics is given in Algorithm \ref{alg:ABM}.

\begin{algorithm}[H]
  \caption{Velocity Verlet integrator for adaptively biased Langevin dynamics with atomic masses $\textbf{M}$, coordinates $\textbf{x}(t)$, momenta $\textbf{p}(t)$, potential $V(\textbf{x}(t))$, forces $F(\textbf{x}(t))$ and friction coefficient $\gamma$,}
  \label{alg:ABM}
    \begin{algorithmic}
      \WHILE{$t < t_{end}$}
        \STATE
            \STATE $\textbf{p}(t+\frac{1}{2}\Delta t) \leftarrow \textbf{p}(t) + \frac{1}{2} \bigl(F(\textbf{x}(t))dt-\gamma \textbf{M}^{-1}\textbf{p}(t) dt + \sqrt{2\gamma\beta^{-1}}dW_t \bigr)$
        \STATE /* Get momenta at half time step
        \STATE
            \STATE $\textbf{x}(t+\Delta t) \leftarrow \textbf{x}(t) + \frac{2}{2+\gamma dt}\textbf{M}^{-1} \textbf{p}(t+\frac{1}{2}\Delta t) dt$
        \STATE /* Propagate coordinates
        \STATE
        \STATE $F(\textbf{x},t+\Delta t) \leftarrow - \nabla V(\textbf{x}(t+\Delta t))$
        \STATE /* get QM or MM forces $F(\textbf{x}(t))$ for new coordinates
        \STATE
        \STATE $F(\textbf{x},t+\Delta t) \leftarrow F(\textbf{x},t+\Delta t) + F^{bias}(\textbf{x},t+\Delta t)$
        \STATE /* call adaptive biasing routine to update bias force
        \STATE
            \STATE $\textbf{p}(t+\Delta t) = \frac{2 - \gamma dt}{2+\gamma dt} \textbf{p}(t+\frac{1}{2}\Delta t) - \frac{1}{2} \bigl(F(\textbf{x}(t+\Delta t))dt-\gamma \textbf{M}^{-1}\textbf{p}(t+\frac{1}{2}\Delta t)) dt + \sqrt{2\gamma\beta^{-1}}dW\bigr)$
        \STATE /* Get momenta at full time step
        \STATE
      \ENDWHILE
    \end{algorithmic}
\end{algorithm}

\subsection{Well-Tempered Metadynamics (WT-MetaD)}
\label{sec:metaD}

MetaD biases a systems dynamic towards undersampled regions along the reaction coordinate $\xi(\textbf{x})$, by accumulating repulsive potentials in regions that have already been visited. The bias potential is typically build by a superposition of repulsive Gaussian kernels and can be written:\autocite{barducci2011metadynamics}
\begin{equation}
  V^{metaD}(\xi,t)= \sum_{k<\frac{t}{\tau_G}} \tau_G \omega \exp\biggr(-\sum_{i=1}^{N_{dim}} \frac{1}{2\sigma_{i}^{2}} (\xi_{i}(\textbf{x})-\xi_{i}(\textbf{x},t_k))^2 \biggl)
\end{equation}
with deposition rate $\tau_G$, Gaussian height $\omega=W/\tau_G$ and variance $\sigma^2$ as free input parameters. In practice $V^{metaD}(\xi,t)$ is stored on a grid and updated every $\tau_G$ time steps for computational efficiency. Over the course of a simulation the bias potential fills local minima along the reaction coordinate until the systems evolution finally resembles a Brownian motion along the flattened free energy surface. The converged bias potential provides an unbiased estimate of the underlying free energy surface
\begin{equation}
  V^{metaD}(\xi, t \to \infty) = - A(\xi) + C
\end{equation}
To avoid oscillation of $V^{metaD}$ around the correct free energy, Well-Tempered metadynamics (WTM) introduces an additional scaling factor of the Gaussian height:\autocite{barducci2008well}
\begin{equation}
  \omega(\xi,t) = \frac{W}{\tau_G}\exp\biggl(-\frac{V^{WTM}(\xi,t)}{k_B \Delta T} \biggr)
\end{equation}
ensuring an decrease of $\omega$ over time and smooth convergence of the Well-Tempered bias potential $V^{WTM}(\xi,t)$. However, the new bias potential does not fully compensate the free energy surface, but can be controlled by parameter $\Delta T$. For $T \to 0$ the bias is zero and ordinary MD is recovered, whereas the limit $\Delta T \to \infty$ corresponds to normal metaD. To obtain a unbiased free energy estimate from $V^{WTM}(\xi,t)$ it has to be scaled accordingly:
\begin{equation}
A(\xi) = -\frac{T+\Delta T}{\Delta T}V^{WTM}(\xi, t)
\end{equation}

\subsection{Adaptive Biasing Force Method (ABF)}
\label{sec:ABF}

The intuition behind ABF is, that adding a force $A'(\xi(\textbf{x}))\nabla \xi(\textbf{x})$ that exactly compensates the average of the original force $-\nabla V(\textbf{x})$ along a given coordinate would result in uniform sampling along this coordinate.\autocite{comer2015adaptive}
Historically, this idea emerged from thermodynamic integration (TI), were the free energy derivative  is computed as the ensemble average of the instantaneous force, $F$, acting along a given reaction coordinate $\xi: \mathbb{R} ^{3N} \to \mathbb{R}$:
\begin{equation}
\frac{dA}{d\xi} = -\braket{F}_{\xi}
\end{equation}
and the free energy is calculated as the integral over this force.\autocite{kirkwood1935statistical,zwanzig1954high}
In practice, as one has no prior knowledge of the free energy derivative, ABF uses an on-the-fly estimate of the mean force acting along the reaction coordinate. For this purpose the transition coordinate $\xi$, connecting two end points, is divided in $M$ equally spaced bins. The approximation of the bias force $\overline{F}(N_{Step},k)$ in bin $k$ is than the average of all collected force samples:\autocite{comer2015adaptive}
\begin{equation}
  \overline{F}(N_{Step},k) = \frac{R(N_{Step}^k,k)}{N_{Step}^{k}} \sum_{\mu=1}^{N_{Step}^{k}} F_{\mu}^{k}
  \label{eq:mean force}
\end{equation}
\begin{equation}
  R(N_{step}^k,k)=\left\{\begin{array}{ll} N_{full}, & N_{step}^{k} < N_{full} \\
                                         1,  & N_{step}^{k} \geq  N_{full} \end{array}\right. \label{eq:ramp}
\end{equation}
with the linear ramp function $R(N_{step}^k,k)$ preventing large fluctuations of the force estimate at the beginning of the simulation from driving the system away from equilibrium. The number of samples when the full biasing force is applied, $N_{full}$, and the bin size are the only free parameters that have to be chosen by the user before the simulation. For a sufficiently large number of samples $N_{step}^k$ equation \ref{eq:mean force} approaches the correct average force in each bin and the free energy difference $\Delta A$ can be estimated by the numerically integrating over the force estimates in individual bins:\autocite{comer2015adaptive}
\begin{equation}
  \Delta A_{\xi} = - \sum_{k=1}^{M} \overline{F}(N_{Step},k) \delta \xi
\end{equation}

The last missing ingredient for the ABF method is an explicit expression for the instantaneous force $F_{\xi}$. Carter et al.\autocite{carter1989constrained} gave a first general expression:
\begin{equation}
  F(\xi,\textbf{q}) = -\frac{\partial V(\xi,\textbf{q})}{\partial \xi} + \beta^{-1} \frac{\partial \ln|J(\xi,\textbf{q})|}{\partial\xi} \label{eq:instforce old}
\end{equation}
which depends implicitly on a vector field $\partial x_i / \partial \xi$, hereafter referred to as "inverse gradient" and on an Jacobian correction term purely geometric in origin. The inverse gradient can be thought of as direction along which an infinitesimal change in $\xi$ is propagated in Cartesian coordinates, the complementary coordinates $\textbf{q}$ being kept constant. A major drawback of this formalism is the requirement of an full coordinate transform from Cartesian coordinates ($\textbf{x}$) to generalized coordinates ($\xi$, $\textbf{q}$).

This requirement could be lifted by den Otter\autocite{den2000thermodynamic}, who put forward the breakthrough idea that the change in $\xi$ can be propagated along an arbitrary vector field $\textbf{v}_i$ ($\mathbb{R}^{3N} \to \mathbb{R}^{3N}$), provided it satisfies some orthonormality conditions.
Extended to multidimensional reaction coordinates \textbf{$\xi$} = ($\xi_i$) and in presence of a set of constraints $\sigma_{k}(\textbf{x})=0$ these read:\autocite{ciccotti2005blue}
\begin{equation}
  \textbf{v}_i \cdot \nabla \xi_j = \delta_{ij} \label{eq:cond1}
\end{equation}
\begin{equation}
  \textbf{v}_i \cdot \nabla \sigma_k = 0 \label{eq:cond2}
\end{equation}
If all reaction coordinates $\xi_i$ are orthogonal to one another and to all constraints, $\textbf{v}_i = \nabla \xi_j/|\nabla \xi_j|^2$ is always a valid option.
Otherwise conditions \ref{eq:cond1} and \ref{eq:cond2} can be fulfilled by orthogonalization:\autocite{ciccotti2005blue}
\begin{equation}
  v_i (\textbf{x}) = \frac{Q^i \nabla \xi_i (\textbf{x})}{|Q^i \nabla \xi_i (\textbf{x})|} \label{eq:ortho v}
\end{equation}
with projector $Q^i$ given by the orthonormal basis $\{\hat{n}|_{j}^{i}(\textbf{x})\}_{j\neq i}$ in the subspace spanned by $\{\nabla \xi_j (\textbf{x})|\}_{j\neq i} \cup \{\nabla\sigma_j (\textbf{x})|\}_{j=1,...,M}$:
\begin{equation}
  Q^i = \textbf{1} - \sum_{j \neq i} \hat{n}_{j}^{i}(\textbf{x}) \otimes \hat{n}_{j}^{i}(\textbf{x})
\end{equation}
Replacing the inverse gradient by vectorfield $\textbf{v}_i$, expression \ref{eq:instforce old} finally reduces to:
\begin{equation}
  F(\xi_i,\textbf{x}) = -\nabla V(\textbf{x}) \cdot \textbf{v}_i(\textbf{x}) + \beta^{-1} \nabla \cdot \textbf{v}_i(\textbf{x})
\end{equation}
but still involves the calculation of second derivatives in the form of the divergence of vector fields $\textbf{v}_i$.\autocite{comer2015adaptive} Analytic expressions for bend angles and torsion angles, used in the present work, are given in the appendix. However, especially for torsion angles orthogonalization via equation \ref{eq:ortho v} becomes exceedingly tedious and inpracticel, significantly limiting the applicability of ABF for multidimensional reaction coordinates.

\subsection{extended Adaptive Biasing Force Method (eABF)}
\label{sec:eABF}
To circumvent the technical requirements of ABF for collective variables, namely being orthogonal to one-another and to constraints, Lesage et al.\autocite{lesage2017smoothed} proposed an more flexible approach named eABF.
In eABF the physical system is extended by additional coordinates $\lambda$ with mass $m_{\lambda}$, which are coupled to the reaction coordinates $\xi_i$ with harmonic potentials. The extended system ($\textbf{x}$, $\lambda$) evolves according to Langevin dynamics in the extended potential
\begin{equation}
  V^{ext}(\textbf{x},\lambda_i) = V(\textbf{x}) + \frac{k_i}{2}(\xi_{i}(\textbf{x})-\lambda_i)^2.
\end{equation}
The key intuition behind eABF is, that in the tight coupling limit efficient sampling of $\lambda$ will result in efficient sampling of $\xi$. Therefore, to obtain uniform sampling along $\xi$ biasing of $\lambda$ is sufficient. The inverse gradient is chosen as null for all physical coordinates ($\textbf{x}$) and 1 for $\lambda$. This way constraints \ref{eq:cond1} and \ref{eq:cond2} are always satisfied, which is especially useful for calculations involving a set of non-orthogonal reaction coordinates.
Sampling the extended system gives the following unbiased Boltzmann distribution in $\lambda$:
\begin{equation}
\begin{aligned}
  \rho^k(\lambda) &\propto
  \int \exp \biggl[-\beta \biggl(V(\textbf{x})+\frac{k}{2}(\xi(\textbf{x})-\lambda)^2 \biggr) \biggr] d\textbf{x} \\
  &= \int \exp \biggl[-\beta V(\textbf{x}) - \frac{(\xi(\textbf{x})-\lambda)^2}{2\sigma^2} \biggr] d\textbf{x}
\end{aligned}
\end{equation}
which depends on the force constant $k$ or variance of the Gaussian kernel $\sigma^2=(\beta k)^{-1}$.
The bias on $\lambda$ is the running average over the spring force in $\lambda$-bin k:
\begin{equation}
  \overline{F}(\lambda_{i}, k) = \frac{\partial A^{k}(\lambda_{i})}{\partial \lambda_i} = \frac{1}{N_{Step}^{k}} \sum_{\mu=1}^{N_{Step}^{k}} k(\lambda_{i,\mu}^{k}-\xi_{i,\mu}^{k})
\end{equation}
For small values of $N_{Step}^{k}$ again the linear ramp function $R(N_{Step},k)$ given by equation \ref{eq:ramp} is used. This generates the following biased Boltzmann distribution:
\begin{equation}
  \tilde{\rho}(\lambda) \propto \int \exp \biggl[-\beta V(\textbf{x})-\frac{(\xi(\textbf{x})-\lambda)^2}{2\sigma^2} + A^{k}(\lambda) \biggr) \biggr] d\textbf{x}
\end{equation}
By using $\int \delta(\xi(\textbf{x})-z)dz=1$ and $A^k(\lambda)=-\beta^{-1}\ln\rho^k(\lambda)$ one can obtain the relationship between unbiased and biased z-distributions:
\begin{equation}
  \tilde{\rho}(z) =  \rho(z) \int \frac{\exp \bigl(-\frac{(\lambda-z)^2}{2\sigma^2}\bigr)}
  {\int \exp\bigl(-\frac{(\lambda-z')^2}{2\sigma^2}\bigr)\rho(z')dz'} d\lambda
\end{equation}
For the tight coupling limit (high $k$, low $\sigma$) the unbiased distribution $\rho(z)$ is recovered and eABF recovers the behavior of standard ABF.
In this case $A^k(\lambda)$  approximates the physical free energy $A(z)$ and the $\Delta A_{z}$ can be approximated by integrating over the converged bias forces on $\lambda$, $\overline{F}(\lambda_{i}, k)$, which will be referred to as "naive estimator" (eABF/naive) hereafter.

An asymptotically unbiased estimator of the free energy can be derived by correcting the free energy gradient obtained from the eABF-biased distribution $\tilde{\rho}(z)$ with the average biasing force on z
\begin{equation}
  \frac{\partial A(z)}{\partial z_i} = -\beta^{-1}\frac{\partial \ln \tilde{\rho}(z)}{\partial z_i} + k(\braket{\lambda_i}_{z}-z_{i})
\end{equation}
which is called "Corrected z-averaged restraint" (CZAR) and can be calculated numerically from the time trajectory ($z_i$,$\lambda_i$) in an post-processing step.\autocite{lesage2017smoothed}

\subsection{Meta-eABF}
\label{sec:meta-eABF}

\section{Standard vs geometric free energy}

	\appendix                     % Anhang
	\backmatter

	\pagenumbering{Roman}                    % für letzte Verzeichnisse wieder romanische Nummerierung
	\sectionnumbering{Roman}
	\setcounter{page}{\theromanPagenumber}   % setzt die aktuelle Seitenzahl von vorne für die romanische Nummerierung fest

	\listoftables
	\listoffigures

%    \nocite{*}								 % auch Quellen die nicht verwendet wurden tauchen in Literaturverzeichnis auf
	\inputencoding{utf8}
	\printbibliography[title=Bibliography]   % gibt Literaturverzeichnis aus


\end{document}
